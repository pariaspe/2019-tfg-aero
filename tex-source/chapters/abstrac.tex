\begin{abstract}
    Este trabajo proporciona una visión global del estado del arte de los sistemas de posicionamiento en interiores. Para ello, se recoge una descripción general de los sistemas y se aplica un división en tres clasificaciones ortogonales para una mejor representación del espectro. La taxonomía propuesta para los sistemas distingue entre técnicas, y a su vez métricas y algoritmos, y tecnologías. Las técnicas engloban al conjunto de herramientas abstractas que pueden usarse en varias tecnologías, es decir, en las diferentes formas específicas de usar señales físicas. \\
    Esta distinción entre diferentes clasificaciones trata de presentar de forma sencilla los sistemas existentes y de forma independiente a la aplicación concreta del sistema de posicionamiento. Durante la primera mitad del trabajo se mantiene un nivel de abstracción elevado, obviando en todo momento el uso final del sistema, para presentar una clasificación y comparativa general a cualquier finalidad. \\
    Además, se desarrolla un método comparativo que permite evaluar y seleccionar uno o varios sistemas de posicionamiento para una aplicación específica. Para ello, una serie de propiedades intrínsecas a los sistemas son explicadas y analizadas. Seguidamente, se establece un modelo con estos criterios que permita evaluar las diferentes tecnologías y que permita hallar el sistema que mejor se adapte a un servicio concreto. \\
    Por último, se selecciona una aplicación para el estudio, MAVs de uso recreativo. Sobre este uso, se aplica la taxonomía presentada, y sobre los resultados obtenidos, se aporta un posible sistema solución al problema definido. Esta solución, basada en un sistema híbrido con tecnologías de radio-frecuencia y inerciales, es una aproximación sencilla e introductoria. El sistema no se concreta un sistema-solución real e implementable, simplemente se formulan una serie de esquemas que explican su funcionamiento básico. \\

\end{abstract}

\keywords{Posicionamiento interiores, IPS, métricas señal, algoritmos posicionamiento, tecnologías posicionamiento, taxonomía, drones, MAVs.}