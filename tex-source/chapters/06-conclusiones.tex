\documentclass[../main.tex]{subfiles}

Este capítulo reúne las conclusiones extraídas tras la realización del proyecto y describe
posibles líneas futuras de trabajo que surgen a partir del mismo.

\section{Conclusiones}
Los objetivos propuestos en la Sección \ref{section:objetivos-intro} se han visto ampliamente satisfechos. En concreto, se ha plasmado una base común de clasificación para los sistemas de posicionamiento en interiores durante los Capítulos \ref{chapter:tecnicas} y \ref{chapter:tecnologias}. En todo momento ha estado presente hacer un ejercicio de abstracción para entender mejor el espectro de sistemas, y de esta forma, hacer una representación sencilla y fácilmente entendible del mismo. Como ya se ha explicado, el espectro se ha divido en tres sub-espectros ortogonales, métricas, algoritmos y tecnologías. Con el término ``ortogonal'' se quiere indicar que no existe solapamiento entre los sub-espectros y que una combinación de los tres da como resultado un sistema particular. \\
En segundo lugar, un método comparativo ha sido expuesto durante el Capítulo \ref{chapter:comparativa} y parte del Capítulo \ref{chapter:estudio}. Durante esta parte del trabajo, ha sido necesario extraer una serie de propiedades intrínsecas a los sistemas, los cuales no se pueden entender sin un correcto análisis de las mismas. A partir de las características propias de los sistemas un modelo matemático ha sido formulado, que permite puntuar tales propiedades y así extraer el sistema más valorado. \\
Por último, una solución al problema introducido durante las Secciones \ref{section:probmotiv-intro} y \ref{section:aplicacion} ha sido sugerida durante el Capítulo \ref{chapter:estudio}. Como se han mencionado en diversas ocasiones a lo largo de esta memoria, la solución recomendada en una simple aproximación a la solución real, pues simplemente se busca una introducción a la resolución del problema. Esta solución es obtenida mediante la comparativa previamente explicada y su elección es discutida durante el capítulo. Sobre la solución se ha aportado unos esquemas que proponen una arquitectura hardware y software. \\
En resumen, las conclusiones sobre los objetivos enumerados en la Sección \ref{section:objetivos-intro} son:

\begin{enumerate}
    \item Se ha propuesto una representación y clasificación para los sistemas de posicionamiento en interiores durante los capítulos \ref{chapter:tecnicas} y \ref{chapter:tecnologias}.
    \item Se ha desarrollado un método comparativo en el capítulo \ref{chapter:comparativa}.
    \item Un sistema ha sido propuesto como solución a la aplicación seleccionada gracias a la comparativa desarrollada (cap. \ref{chapter:estudio}).
\end{enumerate}

\section{Líneas futuras}
Este trabajo quiere sentar las bases en la selección de sistemas para aplicaciones que necesiten un posicionamiento en interiores. Un posible trabajo futuro será concretar y desarrollar la aproximación a la solución propuesta. No solo esta solución podría ser explorada, este estudio abre un abanico de soluciones posibles, tanto para la aplicación tratada como cualquier otra aplicación. \\
Por otro lado, también se propone el desarrollo e investigación de nuevos sistemas, bien sea mediante nuevas métricas, algoritmos o tecnologías. Posiblemente la línea más viable será el estudio de nuevos algoritmos que propongan alternativas en el posicionamiento mediante el análisis del entorno (\emph{fingerprinting}). Esto abarca desde nuevos métodos probabilísticos a nuevos métodos en el tratamiento de imagen. \\

En resumen, las potenciales líneas futuras de trabajo propuestas son:

\begin{itemize}
    \item Continuidad del proyecto y concretar la solución para la solución propuesta. Debe materializarse los esquemas de hardware y software propuestos, y comprobar, bien mediante simulación o pruebas reales, si los supuestos se corresponden con los resultados obtenidos.
    \item Explorar y comparar las soluciones propuestas para la aplicación, pues varias alternativas fueron introducidas antes de seleccionar la utilizada en este trabajo.
    \item Uso de la clasificación y comparativa en cualquier otra aplicación que necesite el posicionamiento en interiores. Esta línea de trabajo no se limita a desarrollos en el ámbito de la robótica.
    \item Nuevos sistemas de posicionamiento. En concreto, nuevos algoritmos basados en el análisis de imagen.
\end{itemize}
